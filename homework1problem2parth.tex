\subsection{Part H}

To find the value of $A = A(a,b,c)$ we can borrow the functions written in previous parts and create a list comprehension (so as to avoid a nested for loop with three layers) to create a table of values 

\begin{tabular}{c | c | c | c}
$a$ & $b$ & $c$ & $A(a,b,c)$ \\
\hline
$\vdots$ & $\vdots$ & $\vdots$ & $\vdots$ 
\end{tabular}

For the sake of brevity I have saved the table rather than printed it for this output PDF (as the instructions did not explicitly say to print the table). The interpolation scheme takes three ordered pairs ($(a_{0}, a_{1})$, $(b_{0}, b_{1})$, $(c_{0}, c_{1})$) and uses the scheme described on the trilinear interpolation Wikipedia page such that $A(a,b,c)$ can be interpolated in the appropriate $a$, $b$, and $c$ domains.

\lstinputlisting{homework1problem2parth.py}
