\subsection{Part B}

We are tasked with developing a random number generator, and it is suggested that we use a combination of several methods mentioned in lecture. This implementation begins with subjecting the input to a MLCG routine then performs a \texttt{XOR} bit shift three times and outputs the resulting integer. The resulting list of pseudo-randomly generated integers has the following form: \texttt{[seed, rng(seed), rng(rng(seed)), \dots]}. The desired list of random numbers should be a set of floating points between 0 and 1 so a second list is generated by dividing each element of the first list by the maximum of the same list.

\lstinputlisting{homework1problem1partb.py}

The first output of this script is the seed: 

\lstinputlisting{homework1problem1partb_print.txt}

To test the quality of this random number generator I plotted the first 1000 points of the list in the fashion described in the assignment instructions (see Figure \ref{fig:11b1}). Each point is assigned a color (ROYGBIV color scheme where the first point is red and the last point is purple) to determine if there is a strong correlation between neighboring points. Qualitatively speaking it looks like points with similar colors are well mixed within the plotted region.

Another test is determining whether all of the floating point values are well sampled in the domain $(0,1]$. Figure \ref{fig:11b2} is a histogram of 1000000 pseudo randomly generated numbers with 20 bins 0.05 wide. There are some small fluctuations but the differences between particular bins are significantly smaller than the total number of numbers within a particular bin, which is encouraging.
